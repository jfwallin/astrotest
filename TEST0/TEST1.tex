\documentclass[11pt]{article}
\usepackage{graphicx}
\usepackage[export]{adjustbox}
\usepackage{multicol}
\marginparwidth0.0in
\evensidemargin 0.0in
\oddsidemargin 0.0in
\textwidth 6.67in
\setlength{\topmargin}{0.0in}
\setlength{\textheight}{9.2in}
\pagestyle{myheadings}
\begin{document}
\pagenumbering{arabic}
\pagebreak 
\setcounter{page}{1}
\vskip 0.1in 

\begin{center}ASTR 1030 - FALL 2017 - EXAM \#7 - WALLIN
\vskip 0.1in 

\center{\LARGE VERSION 1} 
\end{center}
Instructions (Read carefully): 
\begin{enumerate}
\item ABSOLUTELY NO TALKING OR PHONE USE! 
\item {\bf Do not open the exam until you are directed to do so by your instructor!}
\item Write your name, M\#, and your clicker Device ID on the cover sheet below. 
\item Read and sign the Honor Code Certification below.
\item Use your M\# for your ID on the clicker.
\item This is test version 1
\item Read the questions carefully. 
\item Mark all your answers on the paper exam and THEN enter them in your clicker after you have completed the exam with a pen/pencil.
\item When you have completed the exam, turn in the exam to the LA at the front of the room and have your picture ID ready for inspection.
\item GOOD LUCK!!! 
\end{enumerate}
\hrulefill 
\vskip 0.1in 

\begin{itemize} \item Print your name :
\vskip 0.25in 


\item M \# :
\vskip 0.25in 

\item Clicker Device ID : 
\end{itemize} 
\vskip 0.5in 

{\bf Honor Code Certification}
\bigskip

I certify that I have abided by the MTSU honor code in taking this examination. The work
on this exam is my own. I have received no assistance from other persons in completing
this exam. 
\bigskip

Signature:


\pagebreak 

\begin{enumerate}
\setlength{\itemsep}{1pt} 
\setlength{\parskip}{0pt} 
\setlength{\parsep}{0pt}
\setlength{\multicolsep}{1pt} 

\pagebreak 
\begin{minipage}{\textwidth} 
\begin{minipage}{\textwidth} 
\bigskip SECTION = 0
\end{minipage}
\end{minipage}
\begin{minipage}{\textwidth}
\begin{minipage}{\textwidth}
\item Altitude and azimuth measure:
\begin{enumerate} 
\setlength{\itemsep}{1pt} 
\setlength{\parskip}{0pt} 
\setlength{\parsep}{0pt}
\setlength{\multicolsep}{1pt} 
\item Positions on Earth
\item Positions in the sky as seen locally
\item Positions in the sky which are the same for all observers
\end{enumerate} 
\end{minipage}
SECTIONNUMBER=0
\end{minipage}
\vskip 0.20in

\begin{minipage}{\textwidth}
\begin{minipage}{\textwidth}
\item A star's declination is:
\begin{enumerate} 
\setlength{\itemsep}{1pt} 
\setlength{\parskip}{0pt} 
\setlength{\parsep}{0pt}
\setlength{\multicolsep}{1pt} 
\item The angle compass heading of a star measured by an observer.
\item The angle between the star and the horizon of the observer.
\item The distance it is above the Earth's surface.
\item The distance between the star and the Sun.
\item The angle bewteen the star and the Celestial equator.
\end{enumerate} 
\end{minipage}
SECTIONNUMBER=0
\end{minipage}
\vskip 0.20in

\begin{minipage}{\textwidth}
\begin{minipage}{\textwidth}
\item For this question, assume that {\bf Figure 3} shows the position of the Sun in {\bf Murfreesboro} about an hour before Sunset.   Which letter is closest to where the Sun will be in one hour?
\begin{multicols}{3}
\begin{enumerate} 
\setlength{\itemsep}{1pt} 
\setlength{\parskip}{0pt} 
\setlength{\parsep}{0pt}
\setlength{\multicolsep}{1pt} 
\item Position A
\item Position B
\item Position C
\item Position D
\item Position E
\end{enumerate} 
\vfill 
\end{multicols}

\end{minipage}
SECTIONNUMBER=0
\end{minipage}
\vskip 0.20in

\begin{minipage}{\textwidth}
\begin{minipage}{\textwidth}
\item Altitude and azimuth measure:
\begin{enumerate} 
\setlength{\itemsep}{1pt} 
\setlength{\parskip}{0pt} 
\setlength{\parsep}{0pt}
\setlength{\multicolsep}{1pt} 
\item Position of objects in the sky measured from the perspective of a particular observer.
\item Position on Earth.
\item Positions of objects in the sky which are the same for all observers.
\item Positions on the Moon.
\end{enumerate} 
\end{minipage}
SECTIONNUMBER=0
\end{minipage}
\vskip 0.20in

\begin{minipage}{\textwidth}
\begin{minipage}{\textwidth}
\item Latitude and longitude measure:
\begin{enumerate} 
\setlength{\itemsep}{1pt} 
\setlength{\parskip}{0pt} 
\setlength{\parsep}{0pt}
\setlength{\multicolsep}{1pt} 
\item Positions on Earth
\item Positions in the sky as seen locally
\item Positions in the sky which are the same for all observers
\end{enumerate} 
\end{minipage}
SECTIONNUMBER=0
\end{minipage}
\vskip 0.20in

\begin{minipage}{\textwidth}
\begin{minipage}{\textwidth}
\item The stars in a constellation are physically close to one another.
\begin{multicols}{3}
\begin{enumerate} 
\setlength{\itemsep}{1pt} 
\setlength{\parskip}{0pt} 
\setlength{\parsep}{0pt}
\setlength{\multicolsep}{1pt} 
\item True
\item False
\end{enumerate} 
\vfill 
\end{multicols}

\end{minipage}
SECTIONNUMBER=0
\end{minipage}
\vskip 0.20in

\begin{minipage}{\textwidth}
\begin{minipage}{\textwidth}
\item Constellations are close clusters of stars, all at about the same distance from the Sun.
\begin{multicols}{3}
\begin{enumerate} 
\setlength{\itemsep}{1pt} 
\setlength{\parskip}{0pt} 
\setlength{\parsep}{0pt}
\setlength{\multicolsep}{1pt} 
\item True
\item False
\end{enumerate} 
\vfill 
\end{multicols}

\end{minipage}
SECTIONNUMBER=0
\end{minipage}
\vskip 0.20in

\begin{minipage}{\textwidth}
\begin{minipage}{\textwidth}
\item In {\bf Figure 1} at the back of the test, which letter is closest to the constellation of  Cassiopeia?
\begin{multicols}{3}
\begin{enumerate} 
\setlength{\itemsep}{1pt} 
\setlength{\parskip}{0pt} 
\setlength{\parsep}{0pt}
\setlength{\multicolsep}{1pt} 
\item A
\item B
\item C
\item D
\item E
\end{enumerate} 
\vfill 
\end{multicols}

\end{minipage}
SECTIONNUMBER=2
\end{minipage}
\vskip 0.20in

\begin{minipage}{\textwidth}
\begin{minipage}{\textwidth}
\item Where will the Sun be in two hours?
\begin{multicols}{3}
\begin{enumerate} 
\setlength{\itemsep}{1pt} 
\setlength{\parskip}{0pt} 
\setlength{\parsep}{0pt}
\setlength{\multicolsep}{1pt} 
\item A.
\item B
\item C
\item D
\item E
\end{enumerate} 
\vfill 
\end{multicols}

\end{minipage}
SECTIONNUMBER=0
\end{minipage}
\vskip 0.20in

\begin{minipage}{\textwidth}
\begin{minipage}{\textwidth}
\item How many constellations cover surface of the Celestial Sphere?
\begin{multicols}{3}
\begin{enumerate} 
\setlength{\itemsep}{1pt} 
\setlength{\parskip}{0pt} 
\setlength{\parsep}{0pt}
\setlength{\multicolsep}{1pt} 
\item 45
\item 88
\item 120
\item No one knows.
\end{enumerate} 
\vfill 
\end{multicols}

\end{minipage}
SECTIONNUMBER=0
\end{minipage}
\vskip 0.20in

\begin{minipage}{\textwidth}
\begin{minipage}{\textwidth}
\item In figure 1 at the back of the test, which letter is closest to the constellation of Cepheus?
\begin{multicols}{3}
\begin{enumerate} 
\setlength{\itemsep}{1pt} 
\setlength{\parskip}{0pt} 
\setlength{\parsep}{0pt}
\setlength{\multicolsep}{1pt} 
\item A
\item B
\item C
\item D
\item E
\end{enumerate} 
\vfill 
\end{multicols}

\end{minipage}
SECTIONNUMBER=0
\end{minipage}
\vskip 0.20in

\begin{minipage}{\textwidth}
\begin{minipage}{\textwidth}
\item Right ascension and declination measure:
\begin{enumerate} 
\setlength{\itemsep}{1pt} 
\setlength{\parskip}{0pt} 
\setlength{\parsep}{0pt}
\setlength{\multicolsep}{1pt} 
\item Positions on Earth
\item Positions in the sky as seen locally
\item Positions in the sky which are the same for all observers
\end{enumerate} 
\end{minipage}
SECTIONNUMBER=0
\end{minipage}
\vskip 0.20in

\begin{minipage}{\textwidth}
\begin{minipage}{\textwidth}
\item In {\bf Figure 1} at the back of the test, which letter is closest to the constellation of  Cassiopeia?
\begin{multicols}{3}
\begin{enumerate} 
\setlength{\itemsep}{1pt} 
\setlength{\parskip}{0pt} 
\setlength{\parsep}{0pt}
\setlength{\multicolsep}{1pt} 
\item A
\item B
\item C
\item D
\item E
\end{enumerate} 
\vfill 
\end{multicols}

\end{minipage}
SECTIONNUMBER=0
\end{minipage}
\vskip 0.20in

\begin{minipage}{\textwidth}
\begin{minipage}{\textwidth}
\item Right ascension and declination measure:
\begin{enumerate} 
\setlength{\itemsep}{1pt} 
\setlength{\parskip}{0pt} 
\setlength{\parsep}{0pt}
\setlength{\multicolsep}{1pt} 
\item Positions on the Moon.
\item Position of objects in the sky measured from the perspective of a particular observer.
\item Positions of objects in the sky which are the same for all observers.
\item Position on Earth.
\end{enumerate} 
\end{minipage}
SECTIONNUMBER=0
\end{minipage}
\vskip 0.20in

\begin{minipage}{\textwidth}
\begin{minipage}{\textwidth}
\item A star's azimuth is:
\begin{enumerate} 
\setlength{\itemsep}{1pt} 
\setlength{\parskip}{0pt} 
\setlength{\parsep}{0pt}
\setlength{\multicolsep}{1pt} 
\item The distance it is above the Earth's surface.
\item The distance between the star and the Sun.
\item The angle bewteen the star and the Celestial equator.
\item The angle between the star and the horizon of the observer.
\item The angle compass heading of a star measured by an observer.
\end{enumerate} 
\end{minipage}
SECTIONNUMBER=0
\end{minipage}
\vskip 0.20in

\begin{minipage}{\textwidth}
\begin{minipage}{\textwidth}
\item From the horizon to the observer's zenith is an angle of...
\begin{multicols}{3}
\begin{enumerate} 
\setlength{\itemsep}{1pt} 
\setlength{\parskip}{0pt} 
\setlength{\parsep}{0pt}
\setlength{\multicolsep}{1pt} 
\item Azimuth
\item Right Ascension
\item Latitude
\item Declination
\item Altitude
\end{enumerate} 
\vfill 
\end{multicols}

\end{minipage}
SECTIONNUMBER=0
\end{minipage}
\vskip 0.20in

\begin{minipage}{\textwidth}
\begin{minipage}{\textwidth}
\item A star's altitude is:
\begin{enumerate} 
\setlength{\itemsep}{1pt} 
\setlength{\parskip}{0pt} 
\setlength{\parsep}{0pt}
\setlength{\multicolsep}{1pt} 
\item The angle between the star and the horizon of the observer.
\item The angle compass heading of a star measured by an observer.
\item The distance between the star and the Sun.
\item The distance it is above the Earth's surface.
\item The angle between the star and the Celestial equator.
\end{enumerate} 
\end{minipage}
SECTIONNUMBER=0
\end{minipage}
\vskip 0.20in

\begin{minipage}{\textwidth}
\begin{minipage}{\textwidth}
\item In {\bf Figure 1} at the back of the test, which letter is closest to the constellation of Draco?
\begin{multicols}{3}
\begin{enumerate} 
\setlength{\itemsep}{1pt} 
\setlength{\parskip}{0pt} 
\setlength{\parsep}{0pt}
\setlength{\multicolsep}{1pt} 
\item A
\item B
\item C
\item D
\item E
\end{enumerate} 
\vfill 
\end{multicols}

\end{minipage}
SECTIONNUMBER=1
\end{minipage}
\vskip 0.20in

\begin{minipage}{\textwidth}
\begin{minipage}{\textwidth}
\item The closest terrestrial analog to hours of right ascension is angle of longitude.
\begin{multicols}{3}
\begin{enumerate} 
\setlength{\itemsep}{1pt} 
\setlength{\parskip}{0pt} 
\setlength{\parsep}{0pt}
\setlength{\multicolsep}{1pt} 
\item True
\item False
\end{enumerate} 
\vfill 
\end{multicols}

\end{minipage}
SECTIONNUMBER=0
\end{minipage}
\vskip 0.20in

\begin{minipage}{\textwidth}
\begin{minipage}{\textwidth}
\item The celestial sphere is divided into 88 modern constellations.
\begin{multicols}{3}
\begin{enumerate} 
\setlength{\itemsep}{1pt} 
\setlength{\parskip}{0pt} 
\setlength{\parsep}{0pt}
\setlength{\multicolsep}{1pt} 
\item True
\item False
\end{enumerate} 
\vfill 
\end{multicols}

\end{minipage}
SECTIONNUMBER=0
\end{minipage}
\vskip 0.20in

\pagebreak 
\begin{minipage}{\textwidth} 
\begin{minipage}{\textwidth} 
\bigskip SECTION = 1
\end{minipage}
\end{minipage}
\begin{minipage}{\textwidth}
\begin{minipage}{\textwidth}
\item Altitude and azimuth measure:
\begin{enumerate} 
\setlength{\itemsep}{1pt} 
\setlength{\parskip}{0pt} 
\setlength{\parsep}{0pt}
\setlength{\multicolsep}{1pt} 
\item Positions on Earth
\item Positions in the sky as seen locally
\item Positions in the sky which are the same for all observers
\end{enumerate} 
\end{minipage}
SECTIONNUMBER=0
\end{minipage}
\vskip 0.20in

\begin{minipage}{\textwidth}
\begin{minipage}{\textwidth}
\item A star's declination is:
\begin{enumerate} 
\setlength{\itemsep}{1pt} 
\setlength{\parskip}{0pt} 
\setlength{\parsep}{0pt}
\setlength{\multicolsep}{1pt} 
\item The angle compass heading of a star measured by an observer.
\item The angle between the star and the horizon of the observer.
\item The distance it is above the Earth's surface.
\item The distance between the star and the Sun.
\item The angle bewteen the star and the Celestial equator.
\end{enumerate} 
\end{minipage}
SECTIONNUMBER=0
\end{minipage}
\vskip 0.20in

\begin{minipage}{\textwidth}
\begin{minipage}{\textwidth}
\item For this question, assume that {\bf Figure 3} shows the position of the Sun in {\bf Murfreesboro} about an hour before Sunset.   Which letter is closest to where the Sun will be in one hour?
\begin{multicols}{3}
\begin{enumerate} 
\setlength{\itemsep}{1pt} 
\setlength{\parskip}{0pt} 
\setlength{\parsep}{0pt}
\setlength{\multicolsep}{1pt} 
\item Position A
\item Position B
\item Position C
\item Position D
\item Position E
\end{enumerate} 
\vfill 
\end{multicols}

\end{minipage}
SECTIONNUMBER=0
\end{minipage}
\vskip 0.20in

\begin{minipage}{\textwidth}
\begin{minipage}{\textwidth}
\item Altitude and azimuth measure:
\begin{enumerate} 
\setlength{\itemsep}{1pt} 
\setlength{\parskip}{0pt} 
\setlength{\parsep}{0pt}
\setlength{\multicolsep}{1pt} 
\item Position of objects in the sky measured from the perspective of a particular observer.
\item Position on Earth.
\item Positions of objects in the sky which are the same for all observers.
\item Positions on the Moon.
\end{enumerate} 
\end{minipage}
SECTIONNUMBER=0
\end{minipage}
\vskip 0.20in

\begin{minipage}{\textwidth}
\begin{minipage}{\textwidth}
\item Latitude and longitude measure:
\begin{enumerate} 
\setlength{\itemsep}{1pt} 
\setlength{\parskip}{0pt} 
\setlength{\parsep}{0pt}
\setlength{\multicolsep}{1pt} 
\item Positions on Earth
\item Positions in the sky as seen locally
\item Positions in the sky which are the same for all observers
\end{enumerate} 
\end{minipage}
SECTIONNUMBER=0
\end{minipage}
\vskip 0.20in

\begin{minipage}{\textwidth}
\begin{minipage}{\textwidth}
\item The stars in a constellation are physically close to one another.
\begin{multicols}{3}
\begin{enumerate} 
\setlength{\itemsep}{1pt} 
\setlength{\parskip}{0pt} 
\setlength{\parsep}{0pt}
\setlength{\multicolsep}{1pt} 
\item True
\item False
\end{enumerate} 
\vfill 
\end{multicols}

\end{minipage}
SECTIONNUMBER=0
\end{minipage}
\vskip 0.20in

\begin{minipage}{\textwidth}
\begin{minipage}{\textwidth}
\item Constellations are close clusters of stars, all at about the same distance from the Sun.
\begin{multicols}{3}
\begin{enumerate} 
\setlength{\itemsep}{1pt} 
\setlength{\parskip}{0pt} 
\setlength{\parsep}{0pt}
\setlength{\multicolsep}{1pt} 
\item True
\item False
\end{enumerate} 
\vfill 
\end{multicols}

\end{minipage}
SECTIONNUMBER=0
\end{minipage}
\vskip 0.20in

\begin{minipage}{\textwidth}
\begin{minipage}{\textwidth}
\item In {\bf Figure 1} at the back of the test, which letter is closest to the constellation of  Cassiopeia?
\begin{multicols}{3}
\begin{enumerate} 
\setlength{\itemsep}{1pt} 
\setlength{\parskip}{0pt} 
\setlength{\parsep}{0pt}
\setlength{\multicolsep}{1pt} 
\item A
\item B
\item C
\item D
\item E
\end{enumerate} 
\vfill 
\end{multicols}

\end{minipage}
SECTIONNUMBER=2
\end{minipage}
\vskip 0.20in

\begin{minipage}{\textwidth}
\begin{minipage}{\textwidth}
\item Where will the Sun be in two hours?
\begin{multicols}{3}
\begin{enumerate} 
\setlength{\itemsep}{1pt} 
\setlength{\parskip}{0pt} 
\setlength{\parsep}{0pt}
\setlength{\multicolsep}{1pt} 
\item A.
\item B
\item C
\item D
\item E
\end{enumerate} 
\vfill 
\end{multicols}

\end{minipage}
SECTIONNUMBER=0
\end{minipage}
\vskip 0.20in

\begin{minipage}{\textwidth}
\begin{minipage}{\textwidth}
\item How many constellations cover surface of the Celestial Sphere?
\begin{multicols}{3}
\begin{enumerate} 
\setlength{\itemsep}{1pt} 
\setlength{\parskip}{0pt} 
\setlength{\parsep}{0pt}
\setlength{\multicolsep}{1pt} 
\item 45
\item 88
\item 120
\item No one knows.
\end{enumerate} 
\vfill 
\end{multicols}

\end{minipage}
SECTIONNUMBER=0
\end{minipage}
\vskip 0.20in

\begin{minipage}{\textwidth}
\begin{minipage}{\textwidth}
\item In figure 1 at the back of the test, which letter is closest to the constellation of Cepheus?
\begin{multicols}{3}
\begin{enumerate} 
\setlength{\itemsep}{1pt} 
\setlength{\parskip}{0pt} 
\setlength{\parsep}{0pt}
\setlength{\multicolsep}{1pt} 
\item A
\item B
\item C
\item D
\item E
\end{enumerate} 
\vfill 
\end{multicols}

\end{minipage}
SECTIONNUMBER=0
\end{minipage}
\vskip 0.20in

\begin{minipage}{\textwidth}
\begin{minipage}{\textwidth}
\item Right ascension and declination measure:
\begin{enumerate} 
\setlength{\itemsep}{1pt} 
\setlength{\parskip}{0pt} 
\setlength{\parsep}{0pt}
\setlength{\multicolsep}{1pt} 
\item Positions on Earth
\item Positions in the sky as seen locally
\item Positions in the sky which are the same for all observers
\end{enumerate} 
\end{minipage}
SECTIONNUMBER=0
\end{minipage}
\vskip 0.20in

\begin{minipage}{\textwidth}
\begin{minipage}{\textwidth}
\item In {\bf Figure 1} at the back of the test, which letter is closest to the constellation of  Cassiopeia?
\begin{multicols}{3}
\begin{enumerate} 
\setlength{\itemsep}{1pt} 
\setlength{\parskip}{0pt} 
\setlength{\parsep}{0pt}
\setlength{\multicolsep}{1pt} 
\item A
\item B
\item C
\item D
\item E
\end{enumerate} 
\vfill 
\end{multicols}

\end{minipage}
SECTIONNUMBER=0
\end{minipage}
\vskip 0.20in

\begin{minipage}{\textwidth}
\begin{minipage}{\textwidth}
\item Right ascension and declination measure:
\begin{enumerate} 
\setlength{\itemsep}{1pt} 
\setlength{\parskip}{0pt} 
\setlength{\parsep}{0pt}
\setlength{\multicolsep}{1pt} 
\item Positions on the Moon.
\item Position of objects in the sky measured from the perspective of a particular observer.
\item Positions of objects in the sky which are the same for all observers.
\item Position on Earth.
\end{enumerate} 
\end{minipage}
SECTIONNUMBER=0
\end{minipage}
\vskip 0.20in

\begin{minipage}{\textwidth}
\begin{minipage}{\textwidth}
\item A star's azimuth is:
\begin{enumerate} 
\setlength{\itemsep}{1pt} 
\setlength{\parskip}{0pt} 
\setlength{\parsep}{0pt}
\setlength{\multicolsep}{1pt} 
\item The distance it is above the Earth's surface.
\item The distance between the star and the Sun.
\item The angle bewteen the star and the Celestial equator.
\item The angle between the star and the horizon of the observer.
\item The angle compass heading of a star measured by an observer.
\end{enumerate} 
\end{minipage}
SECTIONNUMBER=0
\end{minipage}
\vskip 0.20in

\begin{minipage}{\textwidth}
\begin{minipage}{\textwidth}
\item From the horizon to the observer's zenith is an angle of...
\begin{multicols}{3}
\begin{enumerate} 
\setlength{\itemsep}{1pt} 
\setlength{\parskip}{0pt} 
\setlength{\parsep}{0pt}
\setlength{\multicolsep}{1pt} 
\item Azimuth
\item Right Ascension
\item Latitude
\item Declination
\item Altitude
\end{enumerate} 
\vfill 
\end{multicols}

\end{minipage}
SECTIONNUMBER=0
\end{minipage}
\vskip 0.20in

\begin{minipage}{\textwidth}
\begin{minipage}{\textwidth}
\item A star's altitude is:
\begin{enumerate} 
\setlength{\itemsep}{1pt} 
\setlength{\parskip}{0pt} 
\setlength{\parsep}{0pt}
\setlength{\multicolsep}{1pt} 
\item The angle between the star and the horizon of the observer.
\item The angle compass heading of a star measured by an observer.
\item The distance between the star and the Sun.
\item The distance it is above the Earth's surface.
\item The angle between the star and the Celestial equator.
\end{enumerate} 
\end{minipage}
SECTIONNUMBER=0
\end{minipage}
\vskip 0.20in

\begin{minipage}{\textwidth}
\begin{minipage}{\textwidth}
\item In {\bf Figure 1} at the back of the test, which letter is closest to the constellation of Draco?
\begin{multicols}{3}
\begin{enumerate} 
\setlength{\itemsep}{1pt} 
\setlength{\parskip}{0pt} 
\setlength{\parsep}{0pt}
\setlength{\multicolsep}{1pt} 
\item A
\item B
\item C
\item D
\item E
\end{enumerate} 
\vfill 
\end{multicols}

\end{minipage}
SECTIONNUMBER=1
\end{minipage}
\vskip 0.20in

\begin{minipage}{\textwidth}
\begin{minipage}{\textwidth}
\item The closest terrestrial analog to hours of right ascension is angle of longitude.
\begin{multicols}{3}
\begin{enumerate} 
\setlength{\itemsep}{1pt} 
\setlength{\parskip}{0pt} 
\setlength{\parsep}{0pt}
\setlength{\multicolsep}{1pt} 
\item True
\item False
\end{enumerate} 
\vfill 
\end{multicols}

\end{minipage}
SECTIONNUMBER=0
\end{minipage}
\vskip 0.20in

\begin{minipage}{\textwidth}
\begin{minipage}{\textwidth}
\item The celestial sphere is divided into 88 modern constellations.
\begin{multicols}{3}
\begin{enumerate} 
\setlength{\itemsep}{1pt} 
\setlength{\parskip}{0pt} 
\setlength{\parsep}{0pt}
\setlength{\multicolsep}{1pt} 
\item True
\item False
\end{enumerate} 
\vfill 
\end{multicols}

\end{minipage}
SECTIONNUMBER=0
\end{minipage}
\vskip 0.20in

\pagebreak 
\begin{minipage}{\textwidth} 
\begin{minipage}{\textwidth} 
\bigskip SECTION = 2
\end{minipage}
\end{minipage}
\begin{minipage}{\textwidth}
\begin{minipage}{\textwidth}
\item Altitude and azimuth measure:
\begin{enumerate} 
\setlength{\itemsep}{1pt} 
\setlength{\parskip}{0pt} 
\setlength{\parsep}{0pt}
\setlength{\multicolsep}{1pt} 
\item Positions on Earth
\item Positions in the sky as seen locally
\item Positions in the sky which are the same for all observers
\end{enumerate} 
\end{minipage}
SECTIONNUMBER=0
\end{minipage}
\vskip 0.20in

\begin{minipage}{\textwidth}
\begin{minipage}{\textwidth}
\item A star's declination is:
\begin{enumerate} 
\setlength{\itemsep}{1pt} 
\setlength{\parskip}{0pt} 
\setlength{\parsep}{0pt}
\setlength{\multicolsep}{1pt} 
\item The angle compass heading of a star measured by an observer.
\item The angle between the star and the horizon of the observer.
\item The distance it is above the Earth's surface.
\item The distance between the star and the Sun.
\item The angle bewteen the star and the Celestial equator.
\end{enumerate} 
\end{minipage}
SECTIONNUMBER=0
\end{minipage}
\vskip 0.20in

\begin{minipage}{\textwidth}
\begin{minipage}{\textwidth}
\item For this question, assume that {\bf Figure 3} shows the position of the Sun in {\bf Murfreesboro} about an hour before Sunset.   Which letter is closest to where the Sun will be in one hour?
\begin{multicols}{3}
\begin{enumerate} 
\setlength{\itemsep}{1pt} 
\setlength{\parskip}{0pt} 
\setlength{\parsep}{0pt}
\setlength{\multicolsep}{1pt} 
\item Position A
\item Position B
\item Position C
\item Position D
\item Position E
\end{enumerate} 
\vfill 
\end{multicols}

\end{minipage}
SECTIONNUMBER=0
\end{minipage}
\vskip 0.20in

\begin{minipage}{\textwidth}
\begin{minipage}{\textwidth}
\item Altitude and azimuth measure:
\begin{enumerate} 
\setlength{\itemsep}{1pt} 
\setlength{\parskip}{0pt} 
\setlength{\parsep}{0pt}
\setlength{\multicolsep}{1pt} 
\item Position of objects in the sky measured from the perspective of a particular observer.
\item Position on Earth.
\item Positions of objects in the sky which are the same for all observers.
\item Positions on the Moon.
\end{enumerate} 
\end{minipage}
SECTIONNUMBER=0
\end{minipage}
\vskip 0.20in

\begin{minipage}{\textwidth}
\begin{minipage}{\textwidth}
\item Latitude and longitude measure:
\begin{enumerate} 
\setlength{\itemsep}{1pt} 
\setlength{\parskip}{0pt} 
\setlength{\parsep}{0pt}
\setlength{\multicolsep}{1pt} 
\item Positions on Earth
\item Positions in the sky as seen locally
\item Positions in the sky which are the same for all observers
\end{enumerate} 
\end{minipage}
SECTIONNUMBER=0
\end{minipage}
\vskip 0.20in

\begin{minipage}{\textwidth}
\begin{minipage}{\textwidth}
\item The stars in a constellation are physically close to one another.
\begin{multicols}{3}
\begin{enumerate} 
\setlength{\itemsep}{1pt} 
\setlength{\parskip}{0pt} 
\setlength{\parsep}{0pt}
\setlength{\multicolsep}{1pt} 
\item True
\item False
\end{enumerate} 
\vfill 
\end{multicols}

\end{minipage}
SECTIONNUMBER=0
\end{minipage}
\vskip 0.20in

\begin{minipage}{\textwidth}
\begin{minipage}{\textwidth}
\item Constellations are close clusters of stars, all at about the same distance from the Sun.
\begin{multicols}{3}
\begin{enumerate} 
\setlength{\itemsep}{1pt} 
\setlength{\parskip}{0pt} 
\setlength{\parsep}{0pt}
\setlength{\multicolsep}{1pt} 
\item True
\item False
\end{enumerate} 
\vfill 
\end{multicols}

\end{minipage}
SECTIONNUMBER=0
\end{minipage}
\vskip 0.20in

\begin{minipage}{\textwidth}
\begin{minipage}{\textwidth}
\item In {\bf Figure 1} at the back of the test, which letter is closest to the constellation of  Cassiopeia?
\begin{multicols}{3}
\begin{enumerate} 
\setlength{\itemsep}{1pt} 
\setlength{\parskip}{0pt} 
\setlength{\parsep}{0pt}
\setlength{\multicolsep}{1pt} 
\item A
\item B
\item C
\item D
\item E
\end{enumerate} 
\vfill 
\end{multicols}

\end{minipage}
SECTIONNUMBER=2
\end{minipage}
\vskip 0.20in

\begin{minipage}{\textwidth}
\begin{minipage}{\textwidth}
\item Where will the Sun be in two hours?
\begin{multicols}{3}
\begin{enumerate} 
\setlength{\itemsep}{1pt} 
\setlength{\parskip}{0pt} 
\setlength{\parsep}{0pt}
\setlength{\multicolsep}{1pt} 
\item A.
\item B
\item C
\item D
\item E
\end{enumerate} 
\vfill 
\end{multicols}

\end{minipage}
SECTIONNUMBER=0
\end{minipage}
\vskip 0.20in

\begin{minipage}{\textwidth}
\begin{minipage}{\textwidth}
\item How many constellations cover surface of the Celestial Sphere?
\begin{multicols}{3}
\begin{enumerate} 
\setlength{\itemsep}{1pt} 
\setlength{\parskip}{0pt} 
\setlength{\parsep}{0pt}
\setlength{\multicolsep}{1pt} 
\item 45
\item 88
\item 120
\item No one knows.
\end{enumerate} 
\vfill 
\end{multicols}

\end{minipage}
SECTIONNUMBER=0
\end{minipage}
\vskip 0.20in

\begin{minipage}{\textwidth}
\begin{minipage}{\textwidth}
\item In figure 1 at the back of the test, which letter is closest to the constellation of Cepheus?
\begin{multicols}{3}
\begin{enumerate} 
\setlength{\itemsep}{1pt} 
\setlength{\parskip}{0pt} 
\setlength{\parsep}{0pt}
\setlength{\multicolsep}{1pt} 
\item A
\item B
\item C
\item D
\item E
\end{enumerate} 
\vfill 
\end{multicols}

\end{minipage}
SECTIONNUMBER=0
\end{minipage}
\vskip 0.20in

\begin{minipage}{\textwidth}
\begin{minipage}{\textwidth}
\item Right ascension and declination measure:
\begin{enumerate} 
\setlength{\itemsep}{1pt} 
\setlength{\parskip}{0pt} 
\setlength{\parsep}{0pt}
\setlength{\multicolsep}{1pt} 
\item Positions on Earth
\item Positions in the sky as seen locally
\item Positions in the sky which are the same for all observers
\end{enumerate} 
\end{minipage}
SECTIONNUMBER=0
\end{minipage}
\vskip 0.20in

\begin{minipage}{\textwidth}
\begin{minipage}{\textwidth}
\item In {\bf Figure 1} at the back of the test, which letter is closest to the constellation of  Cassiopeia?
\begin{multicols}{3}
\begin{enumerate} 
\setlength{\itemsep}{1pt} 
\setlength{\parskip}{0pt} 
\setlength{\parsep}{0pt}
\setlength{\multicolsep}{1pt} 
\item A
\item B
\item C
\item D
\item E
\end{enumerate} 
\vfill 
\end{multicols}

\end{minipage}
SECTIONNUMBER=0
\end{minipage}
\vskip 0.20in

\begin{minipage}{\textwidth}
\begin{minipage}{\textwidth}
\item Right ascension and declination measure:
\begin{enumerate} 
\setlength{\itemsep}{1pt} 
\setlength{\parskip}{0pt} 
\setlength{\parsep}{0pt}
\setlength{\multicolsep}{1pt} 
\item Positions on the Moon.
\item Position of objects in the sky measured from the perspective of a particular observer.
\item Positions of objects in the sky which are the same for all observers.
\item Position on Earth.
\end{enumerate} 
\end{minipage}
SECTIONNUMBER=0
\end{minipage}
\vskip 0.20in

\begin{minipage}{\textwidth}
\begin{minipage}{\textwidth}
\item A star's azimuth is:
\begin{enumerate} 
\setlength{\itemsep}{1pt} 
\setlength{\parskip}{0pt} 
\setlength{\parsep}{0pt}
\setlength{\multicolsep}{1pt} 
\item The distance it is above the Earth's surface.
\item The distance between the star and the Sun.
\item The angle bewteen the star and the Celestial equator.
\item The angle between the star and the horizon of the observer.
\item The angle compass heading of a star measured by an observer.
\end{enumerate} 
\end{minipage}
SECTIONNUMBER=0
\end{minipage}
\vskip 0.20in

\begin{minipage}{\textwidth}
\begin{minipage}{\textwidth}
\item From the horizon to the observer's zenith is an angle of...
\begin{multicols}{3}
\begin{enumerate} 
\setlength{\itemsep}{1pt} 
\setlength{\parskip}{0pt} 
\setlength{\parsep}{0pt}
\setlength{\multicolsep}{1pt} 
\item Azimuth
\item Right Ascension
\item Latitude
\item Declination
\item Altitude
\end{enumerate} 
\vfill 
\end{multicols}

\end{minipage}
SECTIONNUMBER=0
\end{minipage}
\vskip 0.20in

\begin{minipage}{\textwidth}
\begin{minipage}{\textwidth}
\item A star's altitude is:
\begin{enumerate} 
\setlength{\itemsep}{1pt} 
\setlength{\parskip}{0pt} 
\setlength{\parsep}{0pt}
\setlength{\multicolsep}{1pt} 
\item The angle between the star and the horizon of the observer.
\item The angle compass heading of a star measured by an observer.
\item The distance between the star and the Sun.
\item The distance it is above the Earth's surface.
\item The angle between the star and the Celestial equator.
\end{enumerate} 
\end{minipage}
SECTIONNUMBER=0
\end{minipage}
\vskip 0.20in

\begin{minipage}{\textwidth}
\begin{minipage}{\textwidth}
\item In {\bf Figure 1} at the back of the test, which letter is closest to the constellation of Draco?
\begin{multicols}{3}
\begin{enumerate} 
\setlength{\itemsep}{1pt} 
\setlength{\parskip}{0pt} 
\setlength{\parsep}{0pt}
\setlength{\multicolsep}{1pt} 
\item A
\item B
\item C
\item D
\item E
\end{enumerate} 
\vfill 
\end{multicols}

\end{minipage}
SECTIONNUMBER=1
\end{minipage}
\vskip 0.20in

\begin{minipage}{\textwidth}
\begin{minipage}{\textwidth}
\item The closest terrestrial analog to hours of right ascension is angle of longitude.
\begin{multicols}{3}
\begin{enumerate} 
\setlength{\itemsep}{1pt} 
\setlength{\parskip}{0pt} 
\setlength{\parsep}{0pt}
\setlength{\multicolsep}{1pt} 
\item True
\item False
\end{enumerate} 
\vfill 
\end{multicols}

\end{minipage}
SECTIONNUMBER=0
\end{minipage}
\vskip 0.20in

\begin{minipage}{\textwidth}
\begin{minipage}{\textwidth}
\item The celestial sphere is divided into 88 modern constellations.
\begin{multicols}{3}
\begin{enumerate} 
\setlength{\itemsep}{1pt} 
\setlength{\parskip}{0pt} 
\setlength{\parsep}{0pt}
\setlength{\multicolsep}{1pt} 
\item True
\item False
\end{enumerate} 
\vfill 
\end{multicols}

\end{minipage}
SECTIONNUMBER=0
\end{minipage}
\vskip 0.20in

\end{enumerate}
\begin{enumerate}
\setlength{\itemsep}{1pt} 
\setlength{\parskip}{0pt} 
\setlength{\parsep}{0pt}
\setlength{\multicolsep}{1pt} 

\begin{multicols}{5} 
\vskip 0.1in 
\item B  
\item E  
\item E  
\item A  
\vskip 0.1in 
\item A  
\item B  
\item B  
\item E  
\vskip 0.1in 
\item E  
\item B  
\item E  
\item C  
\vskip 0.1in 
\item E  
\item C  
\item E  
\item C  
\vskip 0.1in 
\item A  
\item B  
\item A  
\item A  
\end{multicols}

\end{enumerate}
\begin{enumerate}
\setlength{\itemsep}{1pt} 
\setlength{\parskip}{0pt} 
\setlength{\parsep}{0pt}
\setlength{\multicolsep}{1pt} 

\begin{multicols}{5} 
\vskip 0.1in 
\item 13, B  
\item 18, E  
\item 5, E  
\item 3, A  
\item 12, A  
\vskip 0.1in 
\item 8, B  
\item 11, B  
\item 20, E  
\item 15, E  
\item 6, B  
\vskip 0.1in 
\item 2, E  
\item 14, C  
\item 1, E  
\item 4, C  
\item 17, E  
\vskip 0.1in 
\item 9, C  
\item 16, A  
\item 19, B  
\item 7, A  
\item 10, A  
\end{multicols}

\end{enumerate}
\end{document}
